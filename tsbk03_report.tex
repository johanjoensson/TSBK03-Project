\documentclass[a4paper, 12pt]{article}

\usepackage[english]{babel}
\usepackage[utf8x]{inputenc}
\usepackage{amsmath}
\usepackage{graphicx}
\usepackage{csquotes}
\usepackage[hidelinks]{hyperref}

\title{Shadow Volumes using Geometry Shader}
\author{Nora Björklund \and Johan Jönsson}

\begin{document}
\maketitle
\tableofcontents
\newpage
\section{Introduction}
Shadow Volumes is a common way of creating real-time shadows, and has been used
in games such as Doom 3 \cite{gpug1}. In this project the shadow volumes are implemented with the geometry shader. Implementation by geometry shader takes tedious vertex related processing away from the CPU, and insted uses the simultaneous processing power of the GPU.[REF]
\subsection{Project Goal}
The following goal was setup before the start of the project:
\begin{displayquote}
We want to implement shadow volumes in a changable, dynamic scene. Our first priority will be to implement working shadow volumes, and our secondary goal is to implement a game where you get a picture of a shadow and then have to place things vailable so that you get the same shadow as in the picture. We have found a document describing shadow volumes in GPU gems volume 2 and 3. [REF]
\end{displayquote}
\subsection{Planning and Allocation of Work}
(vet ej om denna är nödvändig)
\section{Background}
\subsection{Shadow Volumes}
\subsection{Geometry Shader}
\section{Implementation}
In this chapter we will describe how we implemented the different parts of the
project. We begin by discussing the overall structure of the program and then we
dive into detail on the object loader and how we generate the shadow volumes
using the geometry shader.

\subsection{Program Structure}
We have implemented a z-pass version of the shadow volumes algorithm. We begin
by turning off the colour buffer rendering the scene into the depth buffer and
writing zeros into the stencil buffer (using GL\_NEVER, 0 and G\_REPLACE on
stencil fail). Next we turn off writing to the depth buffer, turn on back face
culling, render the scene and increase the stencil buffer on z-pass. Then we
turn on front face culling and decrement the stencil buffer on z-pass. This way
any part of the scene in shadows will have a non-zero value in the stencil
buffer, we turn the color buffer back on and render all lit parts of the scene
using the stencil buffer (without changing the values in the stencil buffer).
Finally we disable the stencil buffer and turn depth testing back on.

\subsection{Object Loader}
In order to produce proper shadow volumes it is necessary to know the neighbours
of each vertex. Passing this information to the shader is a simple process since
OpenGL has built in support for adjacency primitives. All we need to do is
extract the adjacency information for each object, before we upload them to the
shader. This introduces the need for an object loader capable of extracting this
information at the time of loading the objects in the initialization step.

In order to achieve this it was necessary to write a new object loader, more or
less from scratch (inspiration was of course taken from the functions in
loadobj.~c). For this new object loader we decided to use the open source library
assimp (\url{assimp.sourceforge.net}), because of its simplicity and extensive
capabilities.

A short description of how the object loader works, essentially the first step
is to load the object normally, without adjacency information, just like in
loadobj.~c. Once the object has been loaded we create a simple hash table (the
hash function used is in no way even close to being an optimal one, it is merely
a very simple function that gets the job done reasonably well), in which we
insert every pair of indices in every triangle and their corresponding neighbour
in order to allow us to (hopefully) quickly loop through the entire list of all
triangles in the object in order to find the neighbours of each vertex. Once
this is done we simply loop through every vertex in every triangle and look up
the neighbours from the hash table to create a table of every vertex and its
neighbours in order to set up our list of indices for drawing the object..

\begin{figure}[h]
\centering
\includegraphics[width=10cm, trim = 0mm 50mm 0mm 50mm, clip]{triangle_adj.png}
\caption{The triangle with adjacency primitive. Indices 0, 2, 4 form the
original triangle and indices 1, 3, 5 are the adjacent vertices}
\end{figure}

To clarify what the object loader does, first we loop through every triangle in
the object mesh and create a hash table containing the neighbour information for
all of them. For the triangle (0,1,2) we begin by inserting the edge (0,1) and
the corresponding neighbour (2), then we insert the edge (1,2) and the
neighbour (0) and finally we insert the edge (2,0) and the neighbour (1). By
doing this for every triangle in the entire model mesh we can simply enter the
edges in any triangle and quickly recover the neighbours of each triangle, e.g.
if we input the edge (2,4) we look in our hash table and find the neighbour
(3). Finally we can construct our list of indices, used for drawing the object,
by looping through every vertex in every triangle and looking up the neighbours.

\subsection{Volumes}
To generate the actual shadow volumes in the geometry shader is actually rather
straight forward. We begin by generating the front cap, we simply extract all
triangles facing the light. If the dot product between the light direction and
the triangle normal is positive the triangle faces the light (we take the light
direction to point from the triangle towards the light source). The sides of the
shadow volumes are found by taking each edge and checking whether both triangles
sharing this edge faces the light or not, if one triangle faces the light but
not the other then the edge is part of the side of the shadow volume. We create
the sides of the shadow volumes by creating two triangles (one quad) covering
the area between the edge and its projection at infinity. Projection at infinity
is simple to achieve using homogeneous coordinates, simply put the w-coordinate
to zero.

In this way we can effectively and simply generate the shadow volumes of our
scene using the geometry shader.

\subsection{Geometry Shader syntax}
The geometry shader is a rather recent addition to the OpenGL pipeline and
examples of how to use it are rather scarce. The few examples that do exist are
usually written for older versions of OpenGL, where the geometry shader was an
extension. Therefore in most of the examples found online the syntax used is
both dated and often also unusable in modern OpenGL usage. Something that caused
quite a steep learning curve for using the geometry shader.

[VILL DU HA IN NÅT MER? JAG VET INTE ALLA PROBLEM DU HADE MED DET HÄR]

\section{Results}
Our implementation of shadow volumes is not entirely successful. We have never
been able to render correct shadows on any computer. As can be seen in [LÄMPLIG
FIGUR PÅ KANINER] we lack shadows on the lower rabbit, however the shadows on
the ground seem to have been rendered correctly. The shadow volumes also change
when the camera moves, something that they should definitely not do (unless
there is a light source attached to the camera), unless we run the program on
a computer in Southfork (which unfortunately has a stencil buffer of 0 bytes,
hardly useful for us) where the volumes do not change as the camera moves. The
volumes generated seem to be accurate, when rendered using the stencil buffer,
apart from changing as the camera moves around (again, except for when we use a
Southfork computer). If the camera is kept stationary and the light source
moved, the shadow volumes seem to behave correctly, something that implies that
the volumes computed with the geometry shader are indeed correct.

\section{Discussion}
As was mentioned above, our implementation of shadow volumes contain some
bugs. We have been unable to find out what is causing this really strange
behaviour and have no real ideas left. The error might be because of an error in
the camera and or world matrices, however this seems unlikely as the scenes are
normally rendered correctly, it might be that we multiply the vertices of the
shadow volumes by the wrong combination of these matrices, but we have tried
any combination of them and nothing has solved the problem. In order to really
solve this problem we believe we would have to start over from scratch, simply
redo the entire project and really focus on double checking every step of the
generation of the shadow volumes. This seems like a rather extreme measure, but
we have tried every other possible solution and are currently at our wits' end.

\begin{thebibliography}{9}
\bibitem{gpug1}
	Randima Fernando,
	\emph{GPU Gems: Programming Techniques, Tips and Tricks for Real-Time
	Graphics}.
	Addison-Wesley Professional; First Edition edition, April 1, 2004.
	\url{http://http.developer.nvidia.com/GPUGems/gpugems\_copyrightpg.html}
\end{thebibliography}
\end{document}
